\documentclass{article}
\usepackage[utf8]{inputenc}
\usepackage{amsfonts}
\usepackage{amsmath}
\usepackage[margin=3cm]{geometry}
\usepackage{bbold}
\setlength{\parindent}{0pt}


\title{Modern methods of data analysis \\Homework 3}
\author{Vitaliy Pozdnyakov}
\date{}

\begin{document}

\maketitle

\textbf{Problem 1.} Let $\lambda > 0$ and $X$ be a real-valued random variable with Poisson distribution with parameter $\lambda$:
$\forall k = 0, 1, 2, \dots \text{ } \mathbb P(X = k) = \frac{\lambda^k}{k!}e^{-\lambda}$. Show that, for all $t > 0$, $\mathbb P(X \geq \lambda + t) \leq \exp\left(-\lambda h \left( \frac{t}{\lambda} \right)\right)$, where $h(u) = (1+u)\log(1+u)-u$.

\bigskip

\textbf{Solution.} First, use the Chernoff bound
$$ \mathbb P(X \geq \lambda + t) \leq \frac{\mathbb E \left[ e^{mx} \right]}{e^{m(\lambda + t)}} \text{ for all }  m>0.$$

Next, rewrite the expectation as

$$\mathbb E \left[ e^{mx} \right] = \sum_{k=0}^\infty e^{mk}\frac{\lambda^k}{k!}e^{-\lambda} = e^{-\lambda} \sum_{k=0}^\infty \frac{(e^m\lambda)^k}{k!}.$$

Next, use the property of the exponential function $e^x = \sum_{i=0}^\infty \frac{x^k}{k!}$ and then

$$\mathbb E \left[ e^{mx} \right] = \exp(e^m\lambda - \lambda).$$

Thereby, we obtain

$$ \mathbb P(X \geq \lambda + t) \leq \frac{\exp(e^m\lambda - \lambda)}{\exp(m\lambda + mt)} = \exp[e^m\lambda - \lambda - m(\lambda + t)].$$

From the task description we have $t>0$ and $\lambda > 0$, so $\log \left( \frac{\lambda + t}{\lambda} \right) > 0$ too. Finally, make a substitution $m = \log \left( \frac{\lambda + t}{\lambda} \right)$ that satisfy the Chernoff bound

$$\begin{gathered}
\mathbb P(X \geq \lambda + t) \leq \exp\left[\left( \frac{\lambda + t}{\lambda} \right)\lambda - \lambda - \log \left( \frac{\lambda + t}{\lambda} \right)(\lambda + t)\right]  = \exp\left[ t - \log \left( \frac{\lambda + t}{\lambda} \right)(\lambda + t)\right] \\
= \exp\left[ -\lambda \left( \left(1 + \frac{t}{\lambda}\right) \log \left( 1 + \frac{t}{\lambda} \right) -\frac{t}{\lambda} \right)\right] = \exp\left(-\lambda h \left( \frac{t}{\lambda} \right)\right).
\end{gathered}$$
\bigskip

\textbf{Problem 2.} Let $(\Omega, \mathcal A, \mathbb P)$ be a probability space. Let $Y : \Omega \to \mathbb R$ and $X : \Omega \to \mathbb R^d$ be two random variables.
\bigskip

\textbf{Task 1.} Suppose that $\mathbb E[Y^2] < +\infty$. Show that $\mathbb E[r(X)^2] < +\infty$.

\bigskip

\textbf{Solution.} Rewrite the expression $\mathbb E(r(X)^2)$ via definition of the expectation and regression function

$$\mathbb E[r(X)^2] = \int_{\mathbb R^d} \left( \int_{\mathbb R} y \mathrm{d}k_x(y) \right)^2 \mathrm{d} P_X(x).$$

Next, let $f(X, Y) = Y^2 + 0 \cdot X$. Use the Fubini’s theorem

$$\mathbb E[Y^2] = \mathbb E[f(X,Y)] = \int_{\mathbb R^d} \left( \int_{\mathbb R} y^2 \mathrm{d}k_x(y) \right) \mathrm{d} P_X(x).$$

Now, compare the dissimilar parts of the integrals. Since the function $g(x) = x^2$ is convex we can apply the Jensen's theorem

$$\left( \int_{\mathbb R} y \mathrm{d}k_x(y) \right)^2 \leq \int_{\mathbb R} y^2 \mathrm{d}k_x(y).$$

And thereby,

$$\mathbb E[r(X)^2] \leq \mathbb E[Y^2] < +\infty.$$

\bigskip

\textbf{Task 2.} Suppose that $\mathbb E[Y^2] < +\infty$. Show that, for any $f : \mathbb R^d \to \mathbb R$ such that $\mathbb E[f(X)^2] < +\infty$, we have $\mathbb E[(Y - f(X))^2] = \mathbb E[(Y - r(X))^2] + \mathbb E[(r(X) - f(X))^2]$.
\bigskip

\textbf{Solution.} First, use the Fubini's theorem

$$\mathbb E[(Y - f(X))^2] = \int_{\mathbb R^d}\left(\int_{\mathbb R} (y - f(x))^2 \mathrm{d} k_x(y)\right) \mathrm{d} \mathbb P_X(x).$$

Next, add and subtract the regression function $r(x)$

$$\mathbb E[(Y - f(X))^2] = \int_{\mathbb R^d}\left(\int_{\mathbb R} (y - r(x) + r(x) - f(x))^2 \mathrm{d} k_x(y)\right) \mathrm{d} \mathbb P_X(x).$$

Substitute $a = y - r(x)$ and $b = r(x) - f(x)$ and rewrite the integral with respect to polynomial decomposition $(a - b)^2 = a^2 - 2ab + b^2$

$$\mathbb E[(Y - f(X))^2] = \int_{\mathbb R^d} \int_{\mathbb R} a^2 \mathrm{d}k_x(y) \mathrm{d} \mathbb P_X(x) - 2\int_{\mathbb R^d} \int_{\mathbb R} ab \cdot \mathrm{d}k_x(y) \mathrm{d} \mathbb P_X(x) + \int_{\mathbb R^d} \int_{\mathbb R} b^2 \mathrm{d}k_x(y) \mathrm{d} \mathbb P_X(x).$$

Consider the summands separately. First summand can be rewrited as

$$\int_{\mathbb R^d} \int_{\mathbb R} (y - r(x))^2 \mathrm{d} k_x(y) \mathrm{d} \mathbb P_X(x) = \mathbb E[(Y - r(X))^2].$$

Second summand can be reduced because

$$2\int_{\mathbb R^d} \int_{\mathbb R} (y - r(x))(r(x) - f(x)) \mathrm{d} k_x(y) \mathrm{d} \mathbb P_X(x) = 2\int_{\mathbb R^d} (r(x) - f(x)) \left( \int_{\mathbb R} (y - r(x)) \mathrm{d} k_x(y) \right) \mathrm{d} \mathbb P_X(x) $$

and the expression in the big brackets is

$$\int_{\mathbb R} (y - r(x)) \mathrm{d}k_x(y)  = \int_{\mathbb R} y \cdot \mathrm{d}k_x(y)  - r(x) = 0.$$

Third summand is represented as

$$\int_{\mathbb R^d} \int_{\mathbb R} (r(x) - f(x))^2 \mathrm{d}k_x(y)\mathrm{d}\mathbb P_X(x) = \mathbb E[(r(X) - f(X))^2].$$

Finally, union the summands into the result

$$\mathbb E[(Y - f(X))^2] = \mathbb E[(Y - r(X))^2] + \mathbb E[(r(X) - f(X))^2].$$
\bigskip

\textbf{Task 3.} Suppose that $Y = \varphi(X) + Z$, for some function $\varphi : \mathbb R^d \to \mathbb R$ where $Z$ is independent of $X$ and satisfies $\mathbb E[Z] = 0$ and $\mathbb E[Z^2] = \sigma^2 \in (0, +\infty)$.

\bigskip

\textbf{Subtask 3.a.} Show that $\varphi$ is the regression function of $Y$ given $X$.

\bigskip

\textbf{Solution.} Via another definition of the regression function

$$r(x):= \mathbb E[Y | X = x] = \mathbb E[ \varphi(x) + Z | X = x] = \mathbb E[ \varphi(x)|X = x] + \mathbb E[ Z | X = x].$$

Since $Z$ is independent of $X$ then $\mathbb E[ Z | X = x] = \mathbb E[ Z ] = 0.$ Next, we can consider $\varphi$ as a constant $C_i$ for each given $x_i$ and thereby $\mathbb E[ \varphi(x)|X = x] = \varphi(x)$. Hence, $r(x) = \varphi(x)$.

\bigskip

\textbf{Subtask 3.b.} Show that, for any $f : \mathbb R^d \to \mathbb R$ such that $\mathbb E[f(X)^2] < +\infty$, we have $\mathbb E[(Y - f(X))^2] \geq \sigma^2$.

\bigskip

\textbf{Solution.} In the Task 2 we obtained that

$$\mathbb E[(Y - f(X))^2] = \mathbb E[(Y - r(X))^2] + \mathbb E[(r(X) - f(X))^2].$$

In addition, from the previous subtask we know that $r(x) = \varphi(x)$, so

$$\mathbb E[(Y - f(X))^2] = \mathbb E[(Y - \varphi(X))^2] + \mathbb E[(\varphi(X) - f(X))^2].$$

Next, substitute $\varphi(X) = Y - Z$ in the first brackets in the right side

$$\mathbb E[(Y - f(X))^2] = \mathbb E[Z^2] + \mathbb E[(\varphi(X) - f(X))^2].$$

Obviously $(\varphi(X) - f(X))^2 \geq 0$ and then $\mathbb E[(\varphi(X) - f(X))^2] \geq 0$ too. Thereby

$$\mathbb E[(Y - f(X))^2] \geq \mathbb E[Z^2] = \sigma^2.$$


\end{document}
